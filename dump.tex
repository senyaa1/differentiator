\documentclass[12pt,a4paper]{extreport}
\input{style}
\title{<<Самостоятельная работа по вычислению производных высших порядков>>}
\begin{document}
\maketitle
\pagebreak
\tableofcontents
\pagebreak
\section{Вступление}
В первом классе советской школы математика была не просто предметом, а боевым рубежом. Пока загнивающий Запад в детских садах изучал цвета радуги и делал поделки из макарон, наши первоклассники уже знали, что дифференцировать функции — это не прихоть, а вопрос государственной важности. С урока сразу на доске красовалось грозное: “ДЕРИВАТЫ — старшие братья численных рядов!”. Мелом, быстро и четко. 

Учительница Мария Ивановна, с легким прищуром и неотразимой верой в светлое будущее, объясняла суть производной на примере сбора картошки: “Если Ваня копает одну сотку за 10 минут, а Петя — за 5 минут, то чья производная выше?”. Кто не понимал, оставался после уроков считать частные производные по полям кукурузы.



Зато к концу первой четверти маленькие дифференциаторы могли находить скорость распространения слухов в очереди за колбасой, а на переменах спорили о втором законе Ньютона, пока взрослые стояли в очереди за учебниками. Такие времена, такой уровень. И если кто-то на вопрос “Чему равна производная синуса?” пытался сказать “Что такое синус?”, его тут же отправляли в третий класс — в народное хозяйство стране помощники нужны!\section{Вычисление}


Давайте продифференцируем данную легчайшую функцию.

\begin{dmath*}
f(x) = \sin(x)
\end{dmath*}

Вычислим 1-ую производную:

\begin{dmath*}
f^{(1)}(x) = \cos(x) \cdot 1
\end{dmath*}
Давайте немного упростим данное выражение.


Получаем 1-ую производную:

\begin{dmath*}
f^{(1)}(x) = \cos(x)
\end{dmath*}

Вычислим 2-ую производную:

\begin{dmath*}
f^{(2)}(x) = \sin(x) \cdot 1 \cdot -1
\end{dmath*}
Давайте немного упростим данное выражение.


Получаем 2-ую производную:

\begin{dmath*}
f^{(2)}(x) = \sin(x) \cdot -1
\end{dmath*}

Вычислим 3-ую производную:

\begin{dmath*}
f^{(3)}(x) = \cos(x) \cdot 1 \cdot -1 + \sin(x) \cdot 0
\end{dmath*}
Давайте немного упростим данное выражение.


Получаем 3-ую производную:

\begin{dmath*}
f^{(3)}(x) = \cos(x) \cdot -1
\end{dmath*}

Вычислим 4-ую производную:

\begin{dmath*}
f^{(4)}(x) = \sin(x) \cdot 1 \cdot -1 \cdot -1 + \cos(x) \cdot 0
\end{dmath*}
Давайте немного упростим данное выражение.


Получаем 4-ую производную:

\begin{dmath*}
f^{(4)}(x) = \sin(x)
\end{dmath*}
\section{Заключение}
Наша функция и полученная производная:


\begin{dmath*}
f(x) = \sin(x)
\end{dmath*}

\begin{dmath*}
f^{(4)}(x) = \sin(x)
\end{dmath*}


Несложно заметить, что графики выглядят так:

\begin{minipage}{0.45\textwidth}
\centering\includegraphics[width=\linewidth]{orig_plot.png}\end{minipage}
\hfill
\begin{minipage}{0.45\textwidth}
\centering\includegraphics[width=\linewidth]{optimized_plot.png}\end{minipage}
\end{document}
